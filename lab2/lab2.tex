\documentclass{article}

\usepackage[T1]{fontenc}
\usepackage[utf8]{inputenc}
\usepackage[english]{babel}

\usepackage{amsmath}
\usepackage{siunitx}
\usepackage{hyperref}
\usepackage{listings}
\usepackage{color}
\usepackage{textcomp}
\usepackage{graphicx}

\definecolor{matlabgreen}{RGB}{28,172,0}
\definecolor{matlablilas}{RGB}{170,55,241}

\newcommand{\includecode}[1]{\lstinputlisting[caption={\ttfamily #1.m},
    label={lst:#1}]{matlab/#1.m}}
\newcommand{\inlinecode}[1]{\lstinline[basicstyle=\ttfamily,keywordstyle={}]{#1}}

\author{Riccardo Zanol}
\title{Laboratory 2}

\begin{document}
\lstset{
  language=Matlab,
  basicstyle={\ttfamily \footnotesize},
  breaklines=true,
  morekeywords={true,false,warning,xlim,ylim},
  keywordstyle=\color{blue},
  stringstyle=\color{matlablilas},
  commentstyle={\color{matlabgreen} \itshape},
  numberstyle={\ttfamily \tiny},
  frame=leftline,
  showstringspaces=false,
  numbers=left,
  upquote=true,
}
\maketitle
\section*{Experiment 1}
\section*{Exercise 2}
\begin{enumerate}
\item The histogram of an image is defined as
  \[ p(r_k) = \frac{n_k}{MN} \]
  and is computed using the function \inlinecode{histogram_1} by
  counting the number of pixels with each of the $L=8$ possible
  intensity values and dividing by the total pixel count. The
  resulting histograms for the two images \inlinecode{i1} and
  \inlinecode{i2} are computed in the script \inlinecode{ex2.m}~(lines
  6-21) and are displayed in Fig.~\ref{plot:histo_original}.
  \begin{figure}[h]
    \centering
    \includegraphics[width=0.7\textwidth]{matlab/histo_original}
    \caption{Histograms}
    \label{plot:histo_original}
  \end{figure}
\item To transform the histogram of \inlinecode{i1} so that is is
  equal to that of \inlinecode{i2} it is first necessary to compute
  the two pixel transformations $T(r_k)$ and $G(r_k)$ that equalize
  the histograms of each one of the two images. In
  \inlinecode{ex2.m}~(lines 23-29) these are computed according to the
  equation
  \[ s_k = T(r_k) = (L-1)\sum_{j=0}^kp_r(r_k) . \]
  Then the transformation to get the specified histogram is computed as
  \[ z_k = H(r_k) = G^{-1}(T(r_k)) \]
  where the inversion of $s_k = G(z_k)$ is done by searching for a
  specific $s_k$ in the vector that contains the values of the
  function. If no $s_k$ is equal to the searched value, the $z_k$ that
  maps to the closest value is chosen. In Fig.~\ref{plot:histo_transf}
  the three transformations $T$, $G$ and $H$ are shown, while in
  Fig.~\ref{plot:histo_eq} there is the transformed histogram of
  \inlinecode{i1} compared to its desired shape.
  \begin{figure}[h]
    \centering
    \includegraphics[width=0.7\textwidth]{matlab/histo_transf}
    \caption{Transformations}
    \label{plot:histo_transf}
  \end{figure}
  \begin{figure}[h]
    \centering
    \includegraphics[width=0.7\textwidth]{matlab/histo_eq}
    \caption{Equalized histogram}
    \label{plot:histo_eq}
  \end{figure}  
\end{enumerate}
%\includecode{ex2}
%\includecode{histogram}
\end{document}
