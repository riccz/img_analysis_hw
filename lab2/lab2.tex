\documentclass{article}

\usepackage[T1]{fontenc}
\usepackage[utf8]{inputenc}
\usepackage[english]{babel}

\usepackage{amsmath}
\usepackage{siunitx}
\usepackage{hyperref}
\usepackage{listings}
\usepackage{color}
\usepackage{textcomp}
\usepackage{graphicx}

\definecolor{matlabgreen}{RGB}{28,172,0}
\definecolor{matlablilas}{RGB}{170,55,241}

\newcommand{\includecode}[1]{\lstinputlisting[caption={\ttfamily #1.m},
    label={lst:#1}]{matlab/#1.m}}
\newcommand{\inlinecode}[1]{\lstinline[basicstyle=\ttfamily,keywordstyle={}]{#1}}

\author{Riccardo Zanol}
\title{Laboratory 2}

\begin{document}
\lstset{
  language=Matlab,
  basicstyle={\ttfamily \footnotesize},
  breaklines=true,
  morekeywords={true,false,warning,xlim,ylim},
  keywordstyle=\color{blue},
  stringstyle=\color{matlablilas},
  commentstyle={\color{matlabgreen} \itshape},
  numberstyle={\ttfamily \tiny},
  frame=leftline,
  showstringspaces=false,
  numbers=left,
  upquote=true,
}
\maketitle
\section*{Experiment 1}
\begin{enumerate}
\item To compute the color coordinates in the XYZ colorspace the
  following integral must be computed:
  \[ \beta_k = \int_{350}^{780} C(\lambda)\alpha_k(\lambda) \]
  where the functions $\alpha_k$ are the three color matching
  functions $X(\lambda)$, $Y(\lambda)$ and $Z(\lambda)$ of the
  colorspace. In practice this integral is approximated by computing
  \[ \beta_k = \sum_{\lambda=400}^{780} \frac{c(\lambda)}{w(\lambda)}\frac{d_{65}(\lambda)}{N} \alpha_k(\lambda) \]
  where the sum starts from 400 because the colorimeter's output is
  not very accurate below this wavelength. This equation also takes
  into account the spectra of the white color and of the illuminant
  and the normalization factor $N =
  \sum_{\lambda=400}^{780}d_{65}(\lambda)Y(\lambda)$.

  The script \inlinecode{ex1.m} reads the three CMFs of the XYZ
  colorspace and the spectrum of the light source from the files
  \inlinecode{cmf.mat} and \inlinecode{d65.mat} and interpolates them
  in order to cut them at 400 nm and to have the values of the spectra
  corresponding to the same $\lambda \text{s}$. Then it reads and
  interpolates the white and yellow spectra from
  \inlinecode{macbeth.mat} and it computes the XYZ color coordinates
  of the yellow. These are $(0.631, 0.672, 0.108)$ that correspond to
  $(250, 211, 29.5)$ in the sRGB colorspace. They are close to the
  tabulated values but there is some error.
\item Repeating the same computations of point 1 using the yellow and
  white spectra acquired in the laboratory yields the XYZ color
  coordinates $(0.635, 0.671, 0.142)$. These coordinates correspond in
  the sRGB colorspace to $(250, 211, 62.6)$, that are very close to
  the ones computed at point 1 except for the blue component.
\item To compute the error between the measured and the provided color
  coordinates it is necessary to convert the measured XYZ colors in
  the CIELAB colorspace (or in another space that takes into account
  the McAdam ellipses). The matlab function \inlinecode{xyz2cielab}
  performs the conversion using the equations
  \begin{align*}
    f(t) &= \begin{cases}
      t^{1/3} & \text{if} \quad t>\left(\frac{6}{29}\right)^3 \\
      \frac{1}{3}\left(\frac{29}{6}\right)^2t + \frac{4}{29} \quad & \text{otherwise}
    \end{cases} \\
    L^* &= 116f\left(\frac{Y}{Y_n}\right)-16 \\
    a^* &= 500\left(f\left(\frac{X}{X_n}\right) - f\left(\frac{Y}{Y_n}\right)\right) \\
    b^* &= 200\left(f\left(\frac{Y}{Y_n}\right) - f\left(\frac{Z}{Z_n}\right)\right)
  \end{align*}
  where $(X_n, Y_n, Z_n) = (0.950, 1, 1.09)$ are the XYZ coordinates
  of the reference white.  The euclidean distance computed by
  \inlinecode{ex1.m} between the measured yellow and the tabulated
  values is $7.43$ for the specturm read from
  \inlinecode{macbeth.mat}, while the one read from
  \inlinecode{acquisition_18_3_2016_a.mat} has distance $8.68$.
\end{enumerate}
%\includecode{ex1}
%\includecode{xyz2cielab}
\section*{Exercise 2}
\begin{enumerate}
\item The histogram of an image is defined as
  \[ p(r_k) = \frac{n_k}{MN} \]
  and is computed using the function \inlinecode{histogram_1} by
  counting the number of pixels with each of the $L=8$ possible
  intensity values and dividing by the total pixel count. The
  resulting histograms for the two images \inlinecode{i1} and
  \inlinecode{i2} are computed in the script \inlinecode{ex2.m}~(lines
  6-21) and are displayed in Fig.~\ref{plot:histo_original}.
  \begin{figure}[h]
    \centering
    \includegraphics[width=0.7\textwidth]{matlab/histo_original}
    \caption{Histograms}
    \label{plot:histo_original}
  \end{figure}
\item To transform the histogram of \inlinecode{i1} so that is is
  equal to that of \inlinecode{i2} it is first necessary to compute
  the two pixel transformations $T(r_k)$ and $G(r_k)$ that equalize
  the histograms of each one of the two images. In
  \inlinecode{ex2.m}~(lines 23-29) these are computed according to the
  equation
  \[ s_k = T(r_k) = (L-1)\sum_{j=0}^kp_r(r_k) . \]
  Then the transformation to get the specified histogram is computed as
  \[ z_k = H(r_k) = G^{-1}(T(r_k)) \]
  where the inversion of $s_k = G(z_k)$ is done by searching for a
  specific $s_k$ in the vector that contains the values of the
  function. If no $s_k$ is equal to the searched value, the $z_k$ that
  maps to the closest value is chosen. In Fig.~\ref{plot:histo_transf}
  the three transformations $T$, $G$ and $H$ are shown, while in
  Fig.~\ref{plot:histo_eq} there is the transformed histogram of
  \inlinecode{i1} compared to its desired shape.
  \begin{figure}[h]
    \centering
    \includegraphics[width=0.7\textwidth]{matlab/histo_transf}
    \caption{Transformations}
    \label{plot:histo_transf}
  \end{figure}
  \begin{figure}[h]
    \centering
    \includegraphics[width=0.7\textwidth]{matlab/histo_eq}
    \caption{Equalized histogram}
    \label{plot:histo_eq}
  \end{figure}  
\end{enumerate}
%\includecode{ex2}
%\includecode{histogram}
\end{document}
