\documentclass[a4paper,oneside]{article}

\usepackage[T1]{fontenc}
\usepackage[utf8]{inputenc}
\usepackage[english]{babel}

\usepackage[margin=2.54cm]{geometry}
\usepackage{amsmath}
\usepackage{siunitx}
\usepackage{listings}
\usepackage{color}
\usepackage{textcomp}
\usepackage{graphicx}
\usepackage{subcaption}
\usepackage[section]{placeins}
\usepackage{hyperref}

\definecolor{matlabgreen}{RGB}{28,172,0}
\definecolor{matlablilas}{RGB}{170,55,241}

\newcommand{\includecode}[1]{\lstinputlisting[caption={\ttfamily #1.m},label={lst:#1}]{matlab/#1.m}}
\newcommand{\inlinecode}[1]{\lstinline[basicstyle=\ttfamily,keywordstyle={},stringstyle={},commentstyle={\itshape}]{#1}}

\renewcommand{\vec}[1]{\underline{#1}}
\renewcommand{\Re}[1]{\operatorname{Re}\left[#1\right]}
\newcommand{\E}[1]{\operatorname{E}\left[#1\right]}
\newcommand{\norm}[1]{\left\lVert#1\right\rVert}
\newcommand{\abs}[1]{\left|#1\right|}
\newcommand{\F}[1]{\operatorname{\mathcal{F}}\left[#1\right]}
\newcommand{\ceil}[1]{\left\lceil#1\right\rceil}
\newcommand{\floor}[1]{\left\lfloor#1\right\rfloor}
\newcommand{\Prob}[1]{\operatorname{P}\left[#1\right]}
\newcommand{\ProbC}[2]{\operatorname{P}\left[#1\middle|#2\right]}
\newcommand{\ind}[1]{\operatorname{\mathbbm{1}}\left\{#1\right\}}
\newcommand{\distr}[0]{\sim}
\newcommand{\unif}[1]{\mathcal{U}_{#1}}

\author{Riccardo Zanol}
\title{Laboratory 4}

\begin{document}
\lstset{
  language=Matlab,
  basicstyle={\ttfamily \footnotesize},
  breaklines=true,
  morekeywords={true,false,warning,xlim,ylim},
  keywordstyle=\color{blue},
  stringstyle=\color{matlablilas},
  commentstyle={\color{matlabgreen} \itshape},
  numberstyle={\ttfamily \tiny},
  frame=leftline,
  showstringspaces=false,
  numbers=left,
  upquote=true,
}
\maketitle

To build a panoramic photo it is first necessary to convert the
acquired images in cilindrical coordinates, to do this the provided
function \inlinecode{projectIC} is used. The function converts the
source image to grayscale and then applies the change of coordinates
\begin{align*}
  x' &= r\tan^{-1}\frac{x}{d} \\
  y' &= y\frac{r}{d}\cos\frac{x'}{r}
\end{align*}
point by point, where $d = \frac{W}{2}\frac{1}{\tan\alpha}$ and $r =
\frac{d}{\cos\alpha}$, using half the FOV of the camera as the angle
$\alpha = \SI{33}{\degree}$. One of the converted images is shown in
Fig.~\ref{fig:cil}.
\begin{figure}[htbp]
  \centering
  \includegraphics[width=0.7\textwidth]{matlab/cil_imgs/c6}
  \caption{Image number 6 conerted into cilindrical coordinates}
  \label{fig:cil}
\end{figure}

After all the images have been converted in cilindrical coordinates
their SIFT features and descriptors are computed using the
\inlinecode{sift} function. Then the \inlinecode{match2} function
finds the matching features between two consecutive images by looking
for the closest value of the descriptor and keeping the match when the
distance is closer than \inlinecode{threshold} times the distance form
the second closest descriptor. The value used for the threshold is
$0.7$, and an example of the found features is shown in Fig.~\ref{fig:match}.
\begin{figure}[htbp]
  \centering
  \includegraphics[width=\textwidth]{matlab/saved_match}
  \caption{Matches between images number 6 and number 7}
  \label{fig:match}
\end{figure}
As can be seen the matched features are moslty correct, but there are
some errors on the windows, where there are lots of similar regions,
and some wrong matches on completely unrelated points.

From the matching features the function
\inlinecode{ransac_translation} estimates the translation that
occurred between each pair of consecutive images. This function
randomly selects a pair of matching features, computes the difference
between their coordiantes $(\hat{\Delta x}, \hat{\Delta y})$ and then
counts the feature pairs that are compatible with this translation by
checking wether $\abs{\Delta x_i - \hat{\Delta x}} \leq
\mathit{thresh\_x}$ and $\abs{\Delta y_i - \hat{\Delta y}} \leq
\mathit{thresh\_y}$. The two thresholds chosen are $\mathit{thresh\_x}
= \mathit{thresh\_y} = 5$.


\begin{figure}[htbp]
  \centering
  \includegraphics[width=\textwidth]{matlab/panoramic_img}
  \caption{Panoramic image}
  \label{fig:panoramic}
\end{figure}
\end{document}
