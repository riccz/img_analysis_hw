\documentclass{article}

\usepackage[T1]{fontenc}
\usepackage[utf8]{inputenc}
\usepackage[english]{babel}

\usepackage{amsmath}
\usepackage{siunitx}
\usepackage{hyperref}
\usepackage{listings}
\usepackage{color}
\usepackage{textcomp}
\usepackage{graphicx}
\usepackage{subcaption}
\usepackage[section]{placeins}

\definecolor{matlabgreen}{RGB}{28,172,0}
\definecolor{matlablilas}{RGB}{170,55,241}

\newcommand{\includecode}[1]{\lstinputlisting[caption={\ttfamily #1.m},label={lst:#1}]{matlab/#1.m}}

\newcommand{\inlinecode}[1]{\lstinline[basicstyle=\ttfamily,keywordstyle={}]{#1}}

\author{Riccardo Zanol}
\title{Laboratory 3}

\begin{document}
\lstset{
  language=Matlab,
  basicstyle={\ttfamily \footnotesize},
  breaklines=true,
  morekeywords={true,false,warning,xlim,ylim},
  keywordstyle=\color{blue},
  stringstyle=\color{matlablilas},
  commentstyle={\color{matlabgreen} \itshape},
  numberstyle={\ttfamily \tiny},
  frame=leftline,
  showstringspaces=false,
  numbers=left,
  upquote=true,
}
\maketitle
\section*{Experiment 1}
In the matlab script \inlinecode{ex1.m} each one of the provided
\inlinecode{kodim_*.png} images is mosaicized by
\inlinecode{create_bayer}, that produces an image where the intensity
of each pixel represents only one color component according to the
Bayer pattern ``GRBG'', then the function \inlinecode{demosaic_linear}
is used to linearly interpolate the value of the closest two or four
pixels, depending on the location, to determine the value of each
unknown color component. This function ignores a border of two pixels
around the image and sets it to $(0,0,0)$ to avoid having to handle
this special case in the code.  The script also performs the
demosaicing using the builtin matlab function \inlinecode{demosaic}
which uses a different algorithm, the gradient-corrected bilinear
interpolation, that exploits the correlation between the color
components and corrects the linearly interpolated value of a pixel
with the gradient (of a different color) in that pixel.

After having being demosaicized by each one of the two algortihms the
images are converted into the CIELAB color space to allow them to be
compared to the original images and the mean squared error of the two
algorithms is calculated. The MSE is computed by taking the distance
between each pixel and averaging:
\begin{align*}
  d(i,j) &= \sqrt{\sum_{k=1}^3 \left(A(i,j,k) - B(i,j,k)\right)^2 } \\
  \text{MSE} &= \frac{1}{WH}\sum_{i=0}^H\sum_{j=0}^Wd(i,j)
\end{align*}
where $A(i,j,1) = L(i,j)$, $A(i,j,2) = a(i,j)$ and $A(i,j,3) = b(i,j)$
are the three coordinates in the CIELAB space of the pixel $(i,j)$ of
image $A$. A two pixels wide border is excluded from the comparison
because the function \inlinecode{demosaic_linear} does not perform the
interpolation in this region.  The computed values of the MSE are
shown in Tab.~\ref{tab:mse} where it can be seen that the gradient
correction should improve a lot the quality of the interpolation,
since it halves the MSE.
\begin{table}[h]
  \centering
  \begin{tabular}{ccc}
    Image & Linear interpolation & Gradient corrected interpolation \\
    \hline
    kodim\_01.png & 7.060 & 4.125 \\
    kodim\_05.png & 6.228 & 3.386 \\
    kodim\_13.png & 9.032 & 5.388 \\
    kodim\_19.png & 4.702 & 2.866
  \end{tabular}
  \caption{Mean squared error of the demosaicing algorithms}
  \label{tab:mse}
\end{table}
Unfortunately this is not the case as the gradient correction does
improve a little bit the quality of the reconstructed images, but it
still leaves a lot of noticeable artifacts close to edges and small
details. For example in the detail of image \inlinecode{kodim_19.png},
where both algorithms obtain the smallest value of the MSE, shown in
Fig.~\ref{fig:artifact} there are some colored vertical streaks on the
white fence that are the result of the interpolation.
\begin{figure}[htbp]
  \centering
  \begin{subfigure}{.5\textwidth}
  \centering
  \includegraphics[width=.8\linewidth]{demosaic_kodim19_artifact}
  \caption{\inlinecode{demosaic_linear}}
\end{subfigure}%
\begin{subfigure}{.5\textwidth}
  \centering
  \includegraphics[width=.8\linewidth]{matlab_demosaic_kodim19_artifact}
  \caption{\inlinecode{demosaic}}
\end{subfigure}
\caption{Detail of \inlinecode{kodim_19.png} after the demosaicizing}
\label{fig:artifact}
\end{figure}

\section*{Experiment 2}
\end{document}
