\documentclass[a4paper,oneside]{article}

\usepackage[T1]{fontenc}
\usepackage[utf8]{inputenc}
\usepackage[english]{babel}

\usepackage[margin=2.54cm]{geometry}
\usepackage{amsmath}
\usepackage{siunitx}
\usepackage{listings}
\usepackage{color}
\usepackage{textcomp}
\usepackage{graphicx}
\usepackage{subcaption}
\usepackage[section]{placeins}
\usepackage{hyperref}

\definecolor{matlabgreen}{RGB}{28,172,0}
\definecolor{matlablilas}{RGB}{170,55,241}

\newcommand{\includecode}[1]{\lstinputlisting[caption={\ttfamily #1.m},label={lst:#1}]{matlab/#1.m}}
\newcommand{\inlinecode}[1]{\lstinline[basicstyle=\ttfamily,keywordstyle={},stringstyle={},commentstyle={\itshape}]{#1}}

\renewcommand{\vec}[1]{\underline{#1}}
\renewcommand{\Re}[1]{\operatorname{Re}\left[#1\right]}
\newcommand{\E}[1]{\operatorname{E}\left[#1\right]}
\newcommand{\norm}[1]{\left\lVert#1\right\rVert}
\newcommand{\abs}[1]{\left|#1\right|}
\newcommand{\F}[1]{\operatorname{\mathcal{F}}\left[#1\right]}
\newcommand{\ceil}[1]{\left\lceil#1\right\rceil}
\newcommand{\floor}[1]{\left\lfloor#1\right\rfloor}
\newcommand{\Prob}[1]{\operatorname{P}\left[#1\right]}
\newcommand{\ProbC}[2]{\operatorname{P}\left[#1\middle|#2\right]}
\newcommand{\ind}[1]{\operatorname{\mathbbm{1}}\left\{#1\right\}}
\newcommand{\distr}[0]{\sim}
\newcommand{\unif}[1]{\mathcal{U}_{#1}}

\author{Riccardo Zanol}
\title{Laboratory 5}

\begin{document}
\lstset{
  language=Matlab,
  basicstyle={\ttfamily \footnotesize},
  breaklines=true,
  morekeywords={true,false,warning,xlim,ylim},
  keywordstyle=\color{blue},
  stringstyle=\color{matlablilas},
  commentstyle={\color{matlabgreen} \itshape},
  numberstyle={\ttfamily \tiny},
  frame=leftline,
  showstringspaces=false,
  numbers=left,
  upquote=true,
}
\maketitle
\section*{Fixed threshold}
In this algorithm the moving pixels are determined by computing the
squared difference between the current and the previous frame and
applying a threshold:
\begin{align}
  \rho_k(x,y) &= I_k(x,y) - I_{k-1}(x,y) \\
  m_k(x,y) &= \begin{cases}
    255 \qquad \rho_k^2(x,y) > T_1 \\
    0 \qquad \text{otherwise}
    \end{cases} .
\end{align}
By visualizing the video consisting of the moving pixels $m_k(x,y)$ it
can be seen that this simple threshold algorithm is able to detect the
movement in \inlinecode{chair.avi} without too much noise using a
value of $T_1 = 100$.

The change of focal length at the beginning of the video is detected
as a movement of almost every pixel in the image, but the change in
exposure times do not produce false detections. There are always some
small regions of pixels that are considered moving due to the noise
and the refection in the pane glass is detected but is not
distinguishable from these false movements.  Since this algorithm uses
only the previous frame it stops to detected the chair as a moving
object as soon as it becomes still.  Since this algorithm is unable
to detect movement in uniform regions the pixels that are moving
correspond to the contours of the moving objects.

An example of the detected motion is shown in Fig.\ref{fig:thresh},
for comparison the original frame and its predecessor are shown in
Fig.~\ref{fig:thresh_orig}~and~\ref{fig:thresh_orig_prev}.
\begin{figure}[htbp]
  \centering
  \begin{subfigure}{0.3\textwidth}
    \centering
    \includegraphics[width=0.95\textwidth]{matlab/move_chair_orig_prev}
    \caption{Previous frame}
    \label{fig:thresh_orig_prev}
  \end{subfigure}%
  \begin{subfigure}{0.3\textwidth}
    \centering
    \includegraphics[width=0.95\textwidth]{matlab/move_chair_orig}
    \caption{Current frame}
    \label{fig:thresh_orig}
  \end{subfigure}%
  \begin{subfigure}{0.3\textwidth}
    \centering
    \includegraphics[width=0.95\textwidth]{matlab/move_chair_thresh}
    \caption{Detected motion}
    \label{fig:thresh}
  \end{subfigure}%
  \caption{Example of the threshold algorithm}
\end{figure}

\section*{Background modeling}
To model the background this algorithm keeps a running average of the
frames, computes the squared difference between the current frame and
the background and detects a moving pixel when it's above a threshold:
\begin{align}
  B_k(x,y) &= \alpha I_{k-1}(x,y) + (1-\alpha)B_{k-1}(x,y) \\
  \rho_k(x,y) &= I_k(x,y) - B_k(x,y) \\
  m_k(x,y) &= \begin{cases}
    255 \qquad \rho_k^2(x,y) > T_2 \\
    0 \qquad \text{otherwise}
    \end{cases} .
\end{align}
Trying various values for the learning rate $\alpha \in (0,1)$ it can
be seen that, as it becomes closer to one, the algorithm behaves more
like the simple threshold algorithm because the background weights
more the most recent image instead of the saved background. As
$\alpha$ becomes closer to zero the background incorporates more
slowly the new frames. This causes the detection of some trails behind
the moving objects as can be seen in Fig.\ref{fig:back} and also the
detection of the changes in exposure time as motion. With the
parameters $\alpha = 0.5$ and $T_2 = 130$ we get reasonable
performances, fast movements still leave a trail behind them, but the
change in exposure times are not detected anymore.

\begin{figure}[htbp]
  \centering
  \begin{subfigure}{0.3\textwidth}
    \centering
    \includegraphics[width=0.95\textwidth]{matlab/fast_motion_orig_prev}
    \caption{Previous frame}
    \label{fig:back_orig_prev}
  \end{subfigure}%
  \begin{subfigure}{0.3\textwidth}
    \centering
    \includegraphics[width=0.95\textwidth]{matlab/fast_motion_orig}
    \caption{Current frame}
    \label{fig:back_orig}
  \end{subfigure}%
  \begin{subfigure}{0.3\textwidth}
    \centering
    \includegraphics[width=0.95\textwidth]{matlab/fast_motion_back}
    \caption{Detected motion}
    \label{fig:back}
  \end{subfigure}%
  \caption{Example of the background modeling algorithm}
\end{figure}

\end{document}
