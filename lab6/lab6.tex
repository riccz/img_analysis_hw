\documentclass[a4paper,oneside]{article}

\usepackage[T1]{fontenc}
\usepackage[utf8]{inputenc}
\usepackage[english]{babel}

\usepackage[margin=2.54cm]{geometry}
\usepackage{amsmath}
\usepackage{siunitx}
\usepackage{listings}
\usepackage{color}
\usepackage{textcomp}
\usepackage{graphicx}
\usepackage{subcaption}
\usepackage[section]{placeins}
\usepackage{hyperref}

\definecolor{matlabgreen}{RGB}{28,172,0}
\definecolor{matlablilas}{RGB}{170,55,241}

\newcommand{\includecode}[1]{\lstinputlisting[caption={\ttfamily #1.m},label={lst:#1}]{matlab/#1.m}}
\newcommand{\textinlinecode}[1]{\lstinline[basicstyle=\ttfamily,keywordstyle={},stringstyle={},commentstyle={\itshape}]{#1}}
\newcommand{\inlinecode}[1]{\ifmmode\text{\textinlinecode{#1}}\else\textinlinecode{#1}\fi}

\renewcommand{\vec}[1]{\overline{#1}}
\renewcommand{\Re}[1]{\operatorname{Re}\left[#1\right]}
\newcommand{\E}[1]{\operatorname{E}\left[#1\right]}
\newcommand{\norm}[1]{\left\lVert#1\right\rVert}
\newcommand{\abs}[1]{\left|#1\right|}
\newcommand{\F}[1]{\operatorname{\mathcal{F}}\left[#1\right]}
\newcommand{\ceil}[1]{\left\lceil#1\right\rceil}
\newcommand{\floor}[1]{\left\lfloor#1\right\rfloor}
\newcommand{\Prob}[1]{\operatorname{P}\left[#1\right]}
\newcommand{\ProbC}[2]{\operatorname{P}\left[#1\middle|#2\right]}
\newcommand{\ind}[1]{\operatorname{\mathbbm{1}}\left\{#1\right\}}
\newcommand{\distr}[0]{\sim}
\newcommand{\unif}[1]{\mathcal{U}_{#1}}

\author{Riccardo Zanol}
\title{Laboratory 6}

\begin{document}
\lstset{
  language=Matlab,
  basicstyle={\ttfamily \footnotesize},
  breaklines=true,
  morekeywords={true,false,warning,xlim,ylim},
  keywordstyle=\color{blue},
  stringstyle=\color{matlablilas},
  commentstyle={\color{matlabgreen} \itshape},
  numberstyle={\ttfamily \tiny},
  frame=leftline,
  showstringspaces=false,
  numbers=left,
  upquote=true,
}
\maketitle

To estimate the motion from one frame to the next some features are
first extracted using the Harris corner detector (limiting the number
of returned features to \inlinecode{corner_max_num}), beacuse it would
be very time consuming to run the Lucas-Kanade algorithm on every
pixel in each frame. Then the gradients $I_x$, $I_y$ and $I_t$ are
computed over the whole frame, the spatial ones are computed using the
built-in Matlab function \inlinecode{imgradientxy}, which uses the
Sobel mask by default, while the temporal gradient is approximated
with
\begin{equation}
  I_t(x,y,t) = \frac{I(x,y,t+1) - I(x,y,t-1)}{2} .
\end{equation}

For each feature point $(x_j, y_j)$ a window of $2W+1$ by $2W+1$
pixels is taken from the gradients, centered in $(x_j, y_j)$, so the
optical flow equations for the pixels of the window can be written as
\begin{align}
  \begin{bmatrix}
    I_x(\vec{p_1}) & I_y(\vec{p_1}) \\
    I_x(\vec{p_2}) & I_y(\vec{p_2}) \\
    \vdots & \vdots \\
    I_x(\vec{p_{N_W}}) & I_y(\vec{p_{N_W}}) \\
  \end{bmatrix}
  \begin{bmatrix}
    u \\ v
  \end{bmatrix}
  &= - \begin{bmatrix}
    I_t(\vec{p_1}) \\
    I_t(\vec{p_2}) \\
    \vdots \\
    I_t(\vec{p_{N_W}}) \\
  \end{bmatrix} \\
  A \begin{bmatrix} u \\ v \end{bmatrix} &= - b
\end{align}
where $(u,v)$ is the velocity of the motion that should be estimated
and $\vec{p_i} \in \{ (-W + x_j,\dots W + x_j) \times (-W + y_j,\dots
W + y_j)\}$.  Finally, the motion $(u,v)$ of the window around each
feature point is obtained from the least squares solution:
\begin{align}
  A^TA \begin{bmatrix} u \\ v \end{bmatrix} &= -A^Tb \\
  \begin{bmatrix} u \\ v \end{bmatrix} &= -\left(A^TA\right)^{-1}A^Tb .
\end{align}
The window used in every video has the suggested value for $W=7$.

Since the accuracy of the estimate depends on the eigenvalues of
$A^TA$, which is also the second moment matrix used in the Harris
corner detector, the features are classified according to the
magnitude of the eigenvalues and their ratio in three cases:
\begin{itemize}
  \item when $\lambda_1, \lambda_2 \geq
    \inlinecode{corner_eig_thresh}$ and
    $\frac{\lambda_{max}}{\lambda_{min}} <
    \inlinecode{corner_eigratio_thresh}$ the intensity varies a lot in
    every direction so the feature point is a corner,
    \item when $\lambda_1, \lambda_2 \geq
      \inlinecode{corner_eig_thresh}$ but $\lambda_{max} \geq
      \lambda_{min} \cdot \inlinecode{corner_eigratio_thresh}$ the
      intensity varies a lot olny in one direction so the feature
      point is an edge,
    \item in the other cases the eigenvalues are small so the feature
      point is inside a uniform region.
\end{itemize}
In the output videos, these points and a scaled version of their
motion are drawn in, respectively, green, blue and red. In every
tested video the suggested parameters $ \inlinecode{corner_eig_thresh}
= 10000$ and $ \inlinecode{corner_eigratio_thresh} = 10$ seem to
produce good results.


\end{document}
