\documentclass{article}

\usepackage[T1]{fontenc}
\usepackage[utf8]{inputenc}
\usepackage[english]{babel}

\usepackage{amsmath}
\usepackage{siunitx}
\usepackage{hyperref}
%\usepackage{listings}
\usepackage{color}
%\usepackage{textcomp}
\usepackage{graphicx}

\author{Riccardo Zanol}
\title{Laboratory 1}

\begin{document}
\maketitle
\section*{Experiment 1}
\begin{enumerate}
\item To compute the theoretical resolution from the camera properties it is
possible to apply the formula for the field of view of a thin lense:
\begin{equation}
  \varphi = \tan^{-1}\frac{d}{2f}
  \label{eq:phi}
\end{equation}
where $d$ and $f$ are, respectively, the camera's CCD height and focal
length. Then the height of the portion of plane at distance $l$ that
is projected on the sensor can be obtained with the formula:
\begin{equation*}
  h = 2 l \tan \varphi .
\end{equation*}
Knowing this size, the height of the region that gets projected on a
single pixel can be computed by dividing $h$ by the number of pixels
along the vertical dimension. For the D3300 camera the theoretical
vertical resolutions are \SI{0.961}{\mm} at a focal length of
\SI{18}{\mm} and \SI{0.314}{\mm} at a focal length of \SI{55}{\mm}.
\item Measuring with ``GIMP'' the two lines of the pattern and then
  dividing their known length by their length in pixels (shown in
  Tab.~\ref{tab:pixel-lengths}) gives the theoretical spatial
  resolution of the camera (Tab.~\ref{tab:resolution}). This
  resolution is very close to the one computed from the camera's
  properties.
\begin{table}[h]
  \centering
  \begin{tabular}{rrr}
    Focal length & Vertical line & Horizontal line \\
    \hline
    \SI{18}{\mm} & \SI{163}{px} & \SI{216}{px} \\
    \SI{55}{\mm} & \SI{477}{px} & \SI{635}{px}
  \end{tabular}
  \caption{Measured pixel lengths}
  \label{tab:pixel-lengths}
\end{table}
\begin{table}[h]
  \centering
  \begin{tabular}{rrr}
    Focal length & Vertical resolution & Horizontal resolution \\
    \hline
    \SI{18}{\mm} & \SI{0.926}{\mm} & \SI{0.920}{\mm} \\
    \SI{55}{\mm} & \SI{0.315}{\mm} & \SI{0.314}{\mm}
  \end{tabular}
  \caption{Resolution}
  \label{tab:resolution}
\end{table}
\item Zooming in on the pattern and watching the checkerboard it can
  be observed that the real resolution is less than half of the
  theoretical one. On the picture taken with a focal length of
  \SI{18}{\mm} it is possible to distinguish the \SI{25}{\mm} wide
  checkerboard squares but not the smaller ones. With a greater focal
  length (\SI{55}{\mm}) the squares are distinguishable down to
  \SI{13}{\mm}. This loss of resolution may be caused by noise, by the
  distorsion of the lenses or the camera may not be perfectly focused.
\end{enumerate}
\section*{Experiment 2}
\begin{enumerate}
\item The horizontal and vertical field of view can be computed from
  $f$ using equation (\ref{eq:phi}) where $d$ is the horizontal or
  vertical length of the CCD sensor. For the S9900 camera the field of
  view is:
  \begin{align*}
    \varphi_x &= \tan^{-1}\frac{6.17}{2f} \\
    \varphi_y &= \tan^{-1}\frac{4.55}{2f}
  \end{align*}
  while for the A720is it is:
  \begin{align*}
    \varphi_x &= \tan^{-1}\frac{5.8}{2f} \\
    \varphi_y &= \tan^{-1}\frac{4.35}{2f}
  \end{align*}
\item In the image \texttt{objects\_55mm.jpg} the vertical line is
  641~pixels long, so it corresponds to a line of height $641 \cdot
  h_p$ on the sensor, where $h_p = \frac{15.6}{4000}\,\si{\mm}$ is the
  height of a single pixel on the CCD. Using the equation
\[ \varphi = \tan^{-1} \frac{d}{2f} \]
the angle that the line occupies in the camera's field of view can be
computed. Since the height $h = \SI{20}{\cm}$ of the line is known,
the distance $l$ between the camera and the pattern can be obtained
from the equation
\[ \frac{h}{2} = l \tan \varphi . \]
So the camera is at $l = \frac{hf}{641 h_p} = \SI{4.4}{\m}$ from the
pattern.
\item Let $h_{pix} = \frac{\SI{20}{\cm}}{\SI{641}{px}}$ be the height
  of the region that is projected on a single pixel from distance $l =
  \SI{443.5}{\cm}$. This region is, from the point of view of the
  camera, equivalent to a region positioned at distance $l'$ of height
  $h_{pix}' = \frac{h_{pix}l'}{l}$, since $\tan \varphi =
  \frac{h_{pix}'/2}{l'} = \frac{h_{pix}/2}{l}$. Now that the height
  projected on a pixel from distance $l'$ is known, it is possible to
  compute the size of an object at the same distance as $h_{obj} =
  p_{obj}h_{pix}'$, where $p_{obj}$ is the height of the object in
  pixels. In Tab.~\ref{tab:objects} there are the heights of the three
  objects computed from the image \texttt{objects\_55mm.jpg}.
  \begin{table}[h]
    \centering
    \begin{tabular}{lrrrr}
      Object & $l'$ & $h_{pix}'$ & $p_{obj}$ & $h_{obj}$ \\
      \hline
      Green duck & \SI{252}{\cm} & \SI{0.177}{\mm} & \SI{458}{px} & \SI{8.12}{\cm} \\
      Yellow duck & \SI{352}{\cm} & \SI{0.248}{\mm} & \SI{526}{px} & \SI{13.03}{\cm} \\
      Red bear & \SI{412}{\cm} & \SI{0.290}{\mm} & \SI{425}{px} & \SI{12.32}{\cm}
    \end{tabular}
    \caption{Object sizes}
    \label{tab:objects}
  \end{table}
\end{enumerate}
\section*{Experiment 3}
The three images number 6, 7 and 8 have the same field of view and it
is the narrowest among the pictures, so they must be the ones with $f
= \SI{70}{\mm}$. Looking at the swinging adhesive tape it is possible
to determine the exposure time: the most blurred picture (n. 7) was
taken with the longest one, while the least blurred (n. 8) was shot
with the shortest one. Images number 4 and 5 also have the same focal
length so they are the two images with $f = \SI{24}{\mm}$. The one
with a wider aperture has a shorter depth of field so it is image
number 5 because the vase in the foreground is less focused than in
the other image, while the other objects are the same. Of the three
remaining images, the number 1 has the edges of the table that are the
most parallel, so it is the one with the longest focal length. The
last two images have the same exposure time so they can be
distinguished by their brightness: the one with a wider aperture
(n. 2) is more bright.
\begin{table}
  \centering
  \begin{tabular}{llll}
    Image & Aperture & Exposure time & Focal length \\
    \hline
    Image 1 & f/4.8 & 1/60 s & 35 mm \\
    Image 2 & f/2.8 & 1/60 s & 6 mm \\
    Image 3 & f/3.5 & 1/60 s & 13 mm \\
    Image 4 & f/25 & 1/2 s & 24 mm \\
    Image 5 & f/3.8 & 1/80 s & 24 mm \\
    Image 6 & f/14 & 1/10 s & 70 mm \\
    Image 7 & f/25 & 1/3 s & 70 mm \\
    Image 8 & f/4.5 & 1/80 s & 70 mm
  \end{tabular}
\end{table}
\end{document}
